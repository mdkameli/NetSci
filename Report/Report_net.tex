

\documentclass{article}% use option titlepage to get the title on a page of its own.

\linespread{1.4}

\usepackage{lmodern}
\usepackage{listings}
\usepackage{color}

\definecolor{dkgreen}{rgb}{0,0.6,0}
\definecolor{gray}{rgb}{0.5,0.5,0.5}
\definecolor{mauve}{rgb}{0.58,0,0.82}





\title{%
  Political Manifesto Analysis \\
  \large A network science approach}

\date{2019, January}
\author{Giovanni Misseri, Mohammad Kameli, Jolan Chabbey \\ \\ 
Network Science course}
\begin{document}
\maketitle
\section{Introduction}
In this relation we will try to explain and justify what we've done in order to analyze the european political society.
As one can imagine the political world is dense of alleances, sometimes decleared and sometimes unspoken; and even a bigger problem is the one of understanding the actual political thought of a party.
Usually each party declears itself as lined up on right center or left side, but in practice often it's really hard to define what makes a party a right side or left side party and if we just look at how a certain party acts the definition becomes even more fuzzy.\\

What we aim to achieve with this work is to derive and analize the network of european parties in a cross-temporal analysis. As we said is not always the case that a party that declares itself as a left side party, acts like a left side party. Due to this reason to construct the network in which each party is a node, we will link two parties not if they declare to collaborate each other and nor if they have some political alleance, we will try to link two nodes if the two connected parties have the same idea on some critical concepts.\\

On such network we will run our analysis trying to spot behaviours and insits of the political scene. Moreover we will try to detect the communities inside this network in order to see if the actual division of the parties overlaps with the decleared one, and consequently see which are the parties that, based on their ideas, are not distiguishable by their adversary.

\section{Manifesto Project}

The Manifesto Project - "https://manifesto-project.wzb.eu/" - is a German project which aims to mantain and contiously update different political dataset. Their main work is the "Manifesto Project Dataset", a dataset containing at the moment 4388 political manifestos published from 1945 till today by 1117 parties.












\end{document}